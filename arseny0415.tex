\documentclass{article}
\usepackage[utf8]{inputenc}

% Russian specifics
\usepackage[russian]{babel}

% Math specifics
\usepackage{amsmath}
\usepackage{gensymb}

\title{Math problems}

\begin{document}
\maketitle
\section*{Tier 1}
\textbf{Задача 1.}
В треугольнике $ABC$ угол $\angle C$ прямой, а длина отрезка $AB$ равна $13$. Известно, что $\tg \angle BAC  = 5$. Проведена высота треугольника $CH$. Найдите длину отрезка $HB$.

\textsc{Комментарий: }
В данной задаче необходимо использовать определение тангенса в прямоугольном треугольнике в сочетании с теоремой Пифагора. Обозначая неизвестные длины меременными составить одно или несколько уравнений. Решение полученного уравнения или системы может привести к искомому решению. Приветствуются разные подходы, помните, задача может иметь множество очень разных но верных решений.


\textbf{Задача 2.}
Построить график функции $y = |-3x^2 - |x| + 4|$.

\textsc{Комментарий: }
Хорошо бы вспомнить такие определения как: график функции, четность функции, квадратное уравнение. Необходимо знать, как найти вершину параболы, как определить напрвление ветвей. 
\section*{Tier 2}
\textbf{Задача 3. (ЗФТШ. Геометрия №7)}
Постройте треугольник по стороне и проведённой к ней высоте, если известно, что эта сторона видна из центра вписанной в треугольник окружности под углом $135^{\degree}$.

\textsc{Комментарий: }
Решая задачу, подразумевается, что школьник умеет делать элементарные построения циркулем и линейкой. Необходимо знать сооттношения между вписанными углами в окружность, знать, как построить перпендикуляр из заданной точки к заданной прямой, как построить касательную к окружности из заданной точки.
\end{document}